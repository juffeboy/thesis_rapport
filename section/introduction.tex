\section{Introduction}
%The main paradigm of networks today can be called  \textit{host centric} and most of our communication is defined as end-to-end between these \textit{named hosts}. It is predicted that in 2020 we reach around 50 billion Internet of Things (IoT) devices \cite{alanCarlton} and the usage of these devices often imply information centric usage patterns \cite{Ahlgreniot}.\todo{Hela introt lite väl kort...} \todo{Borde nämna ``named hosts'' här också.}

%Information-centric networking (ICN) is a communication paradigm for the future Internet that is based on \textit{named data} instead of \textit{named hosts}. The communication is defined through requesting and providing named data, decoupling senders from receivers. This could make it possible to integrate storage for caching within the network infrastructure \cite{Ahlgren2012} which could lead to improvements in the network as a whole.


%Information-centric networking (ICN) is a new communication paradigm for the future Internet that is based on \textit{named data} instead of \textit{named hosts}. The communication is defined through requesting and providing named data, which enables the senders to be decoupled from the receivers. 

%The main paradigm of todays network dates back to the early birth 

%\begin{itemize}
%	\item berätta om hur nätverk är idag
%	\item IoT användningen ökar.
%	\item ICN nytt protocol att kommunicera.
%	\item 
%\end{itemize}
%
%\begin{itemize}
%\item natverk idag.
%\end{itemize}
The main paradigm of networks today have evolved around a \textit{host centric} networking model which enables communication between two defined hosts. This communication model emerged with the first computer networks that was established in the 60's. It made it possible for a user to e.g connect and communicate with super computers at the time.
With the introduction of the world wide webb in the middle of the 1990's, the communication model of information changed for the Internet, going from the \textit{host centric} view towards a \textit{data centric} view. With the Internet, it is the information produced by a website or a streaming service that is important, not necessarily the physical location of the content.

Information-centric networking (ICN) is a communication paradigm for the future Internet that is based on \textit{named data} instead of \textit{named hosts}. The communication is defined through requesting and providing named data, decoupling senders from receivers. This could make it possible to integrate storage for caching within the network infrastructure which could lead to improvements in the network as a whole \cite{Ahlgren2012} .

%\begin{itemize}
%\item Iot
%\end{itemize}
It is predicted that in the year of 2020 we will reach around 50 billion Internet of Things (IoT) devices \cite{alanCarlton}. These machines consists of high heterogenety and ranges from wireless sensors, e.g able to produce environment data, to wearables, smart home units and many other types of devices that was not previously connected to the Internet.

One area where these IoT devices differs from regular PCs, phones and similar, is that they have a different view regarding the usage of resources. For IoT units, things such as processing power, memory availability and power capabilities are most often very limited and those constraints are expected to stay within the domain in the future \cite{chipmakersarebetting}.

The usage of these IoT devices, and their data, often implies information centric usage pattern that is similar to the ICN \cite{Ahlgreniot}. Whereas, the content produced by the IoT device is consumed, by an user or application, on the network instead of communicating directly with the producer or host.


%Recently there has been some investigation about the applicability and trade-offs of the ICN paradigm in the IoT/sensor domain \cite{surveyPaper2}. It will be interesting to investigate the mapping of IoT-sensor data directly to \textit{named data} in ICN and therefore use the network as an application-independent distribution for IoT data values. 

\subsection{Problem statement}
%The overal objective with this thesis is to make an experimental evaluation of the performance and applicability of using ICN in the IoT/sensor domain. The goal is to assess the feasibility and performance

The purpose of this thesis project is to evaluate the performance and feasibility of using the ICN paradigm in the IoT domain. The scenarios are normal usage patterns between devices including producing and consuming data continuously created in a periodic interval. 

%By evaluating the performance of ICN components and the application as a whole, one can provide a comparision between such an application and a network reference \todo{vad menas med ref point och resterande del av meningen. // A} point. Ideally it should be possible \todo{Formulera om. // A} to divide and present the throughput, and how much delay time, spent operating under ICN. Therefore a part of the goal of this thesis project is to provide an in depth analysis of where most time is spent when using the ICN application. 
A major goal of this thesis project is to do an in depth analysis regarding the performance and throughput of ICN used in the IoT domain. How long are the latency responses for a consumer and where most of the time is spent using the ICN paradigm are questions that this theses tries to answer. It is to be investigated whether or not ICN perform well enough in order to serve its consumers with data.

ICN is by nature a pull paradigm where a consumer has to initiate a request of a particular data object in order to retrive it. This is in constrast to several IoT systems where the publish/subscribe approach is more common. An example of such a system is MQTT \cite{mqtt} where the producer sends the newly created data towards a broker, which then pushes the data to a user/consumer. %By evaluating the feasibility, a major goal is to investigate if it is possible to achive similar outcome with ICN together with a onetime subscription model and if such a system would be stable. With the onetime subscription model, a consumer tries to pull the data from the producers by initiating a request just before the data has been created. 
By evaluating the feasibility of ICN, a major goal is to investigate if it is possible to distribute data efficiently with ICN and a one time subscription model. It is in the interest of this thesis to determine how stable such a system is and how well it perform over time. With the onetime subscription model, a consumer tries to pull the data from the producers by initiating a request just before the data has been created. 



\subsection{Delimitations}
CCN will be the only ICN architecture covered in this thesis. Alternative implementations such as Psirp, Netinf among others, will not be covered at all.
Furthermore, this thesis will not develop any further functionallity on the current CCN implementation for Contiki-OS. The current implementaion is to be considered sufficient. Exceptions are made for functionallity regarding monitoring and measurement metrics that will have effect on the evaluation. 

Different cache strategies for the local storage at a CCN node and other functionallities that would be desirable, but not necessary, is considered out of the scope of this thesis. Furthermore, no aspect of power consumption will be taken into account for the thesis.

\subsection{Research methodology}
%\todo{Behöver kompletteras med lite mer generell text om experimentell forskning.// B}
%Currently there is a \todo{Berätta att ccn-lite är en lw impl av ccn, sen har yanqiu portat det till contiki.// a} CCN-lite implementation for the IoT light-weighted Contiki OS, the implementation was conducted by a former thesis worker Yanqiu Wu \cite{yanqui}.
Currently there is a light-weight implementation of CCN available on several operating systems called CCN-lite \cite{CCN-LITE}. In a former thesis project, Yanqiu Wu ported the CCN-lite implementation to the IoT operating system Contiki OS \cite{yanqui}. In this project, a qualitative and quantitative performance evaluation will be carried out through experimentation with this implementation. The sensor hardware used is the Texas Instruments SensorTag CC-2650, which runs the CCN-lite implementation. 
%\todo{Short cycels of what? // a}short cycles

Since the thesis is based on experiments a large amount of measurements will be conducted. The methodology used will be  small loops between each measurement. %This iterative approach has the advantage that it can give answers to \todo{Formulera om, vad är det som menas? // a}questions that are not specifically asked and it can also be used to see changes over time. 
In order to make the testing intervals short and feasible, a lot of effort has been put into creating smart scripts that automate testing and representation of the data so it becomes visual to the tester. 

Different software tools will be programmed to measure the desired performance metrics. 
In order to evaluate the performance, various types of measuring tools will be conducted. These tools has either a general or a specific purpose. A general purpose tool can for instance collect the measured data from the test, and make it representable for the user. Specific purpose tools on the other hand can be ones that measure the processing time on a sensor or measure the latency time. 

\subsection{Contributions}

This thesis contribute an experimental evaluation, with performance and feasibility aspects in focus, of using CCN on low power sensor hardware in a typical IoT enviroment.

There are several other contributions of this thesis project. One is improving the current CCN-lite application for Contiki-OS, mostly to make it more stable and reliable when run on low power sensor hardware. Another contribution is the development of a tool that calculates the roundtrip times for IPv6 devices to the CCN-lite project. Moreover, software enabling a consumer to retrieve data from the publisher through a one-time subscription model described in later parts.

Furthermore, several testing tools have been developed to make this thesis possible. Although a lot of time and effort has been consumed into creating these software implementations, the specific details concerning implementation will not be covered in this report. However, some higher abstractation regarding algorithms and logics will be presented.


\subsection{Related work}
When Jacobson et al. publicated the paper \textit{Networking named content} in 2009 it sparked ideas of an alternative approach of communicating in contrast to current IP networking \cite{Jacobson2009}. They implemented a prototype which replaced IP with CCN in the network stack and proved that it could be a potential alternative for the future Internet.\\\\
Since then, several research papers have been published comparing different ICN alternatives and their potential benefits and trade-offs when implemented as a network service\cite{Ahlgren2012}. Studies prove that it could be a suitable replacement of current IP networking structures, but there is a need for more performance analysis and studies\cite{Ahlgren2012}\cite{Greek-ICN-networking-survey-2014}.\\\\
However, it was not until the last couple of years that the research community started to investigate the feasibility and applicability of using ICN in the IoT domain. Several studies, the majority being literature studies or theoretic in nature, have been conducted recently or are currently ongoing. There only seem to be two implementation studies available to date.\\\\
In a conceptual study conducted by Ahlgren et al. \cite{Ahlgreniot}, it is concluded that an advantage with ICN, is that the naming of data is independent of the device that produces it. The decoupling between a producer and consumer of data could improve performance in lossy networks.Challenges reside in the naming of data that is produced periodically over time where a major issue is to retrieve the \textit{latest} value in that sequence. Potential solutions could be to implement a \textit{one-time subscription}, where the request is stored in the cache at the node until the data becomes available\cite{Ahlgreniot}.\\\\
Another study by Amadeo et al. \cite{iotchop} argues ICN is by nature close to the IoT domain. They also conclude that there is a need of further investigations regarding if ICN should be implemented as an overlay of existing IP infrastructure, coexist with IP, or if it should be a replacement in the same manner as proposed in \cite{Jacobson2009}.\\\\
Baccelli et al. were first to port of CCN-lite to the IoT operating system RIOT \cite{icniotexpinwild}\cite{RIOT}. The project was the first trial of implementing ICN without any IP protocol in the IoT domain. They compared their CCN-lite implemententation and a regular 6LowPAN/IPv6/RPL approach and saw that there were several advantages using ICN over IP. Although they identified several areas where further work needs to be done, they argue that ICN is applicable in the IoT domain.\\
Another implementation of CCN for IoT devices was a thesis project conducted by Yanqui Wu at SICS/KTH\cite{yanqui}. He ported the CCN-lite functionallity into another IoT operating system, Contiki OS, and implemented the software as an middle-layer between the application- and the transport layer. Although some evaluation was done, there was no further investigation on how well CCN performs at the application layer. \\

%\subsection{Related work}
%When Jacobson et al. publicated the paper \textit{Networking named content} in 2009 it sparked ideas of an alternative approach of communicating in contrast to current IP %networking \cite{Jacobson2009}. They implemented a prototype which replaced IP with CCN in the network stack and proved that it could be a potential alternative for the %future Internet.\\\\
%Since then, several research papers have been published comparing different ICN alternatives and their potential benefits and trade-offs when implemented as a network service\%cite{Ahlgren2012}. Studies prove that it could be a suitable replacement of current IP networking structures, but there is a need for more performance analysis and studies\%cite{Ahlgren2012}\cite{Greek-ICN-networking-survey-2014}.\\\\
%However, it was not until the last couple of years that the research community started to investigate the feasibility and applicability of using ICN in the IoT domain. %Several studies, the majority being literature studies or theoretic in nature, have been conducted recently or are currently ongoing. There only seem to be two implementation %studies available to date.\\\\
%In a conceptual study conducted by Ahlgren et al. \cite{Ahlgreniot}, it is concluded that an advantage with ICN, is that the naming of data is independent of the device that %produces it. The decoupling between a producer and consumer of data could improve performance in lossy networks\todo{Behöver koppla named data till IoT-context - Namngivet %sensordata // bengt}.Challenges reside in the naming of data that is produced periodically over time where a major issue is to retrieve the \textit{latest} value in that %sequence. \todo{Prata mer om hur man hittar senaste värdet. // anders} Potential solutions could be to implement a \textit{one-time subscription}, where the request is stored %in the cache at the node until the data becomes available\cite{Ahlgreniot}.\\\\
%Another study by Amadeo et al. \cite{iotchop} argues ICN is by nature close to the IoT domain. They also conclude that there is a need of further investigations regarding if %ICN should be implemented as an overlay of existing IP infrastructure, coexist with IP, or if it should be a replacement in the same manner as proposed in \cite{Jacobson2009}.%\\\\
%Baccelli et al. were first to port of CCN-lite to the IoT operating system RIOT \cite{icniotexpinwild}\cite{RIOT}. The project was the first trial of implementing ICN without %any IP protocol in the IoT domain. They compared their CCN-lite implemententation and a regular 6LowPAN/IPv6/RPL approach and saw that there were several advantages using ICN %over IP. Although they identified several areas where further work needs to be done, they argue that ICN is applicable in the IoT domain.\\
%Another implementation of CCN for IoT devices was a thesis project conducted by Yanqui Wu at SICS/KTH\cite{yanqui}. He ported the CCN-lite functionallity into another IoT %operating system, Contiki OS, and implemented the software as an middle-layer between the application- and the transport layer. Although some evaluation was done, there was %no further investigation on how well CCN performs at the application layer. \\
%


\subsection{Structure of the Report}
This report is structured in six sections, omitting the introduction.

Section 2 will discuss background knowledge regarding the problem to this date. Technical details is presented that enable the reader to understand the details for the rest of the report. The background will first cover the network stack focusing on IoT devices, thereafter ICN will be convered generally and CCN in more depth. A breif overview of the Contiki-OS gives the reader knowledge about the OS running on the sensors.

Section 3, implementation, discuss what components are targeted for the evaluation. A in depth description of the problem statement is discussed, aswell as identifying them. It is to provide an adequate evaluation of using ICN in the IoT domain, that certain parts of the CCN implementation is selected. These have been choosen with advice of my mentors, Bengt Ahlgren and Anders Lindgren. Which experiments the thesis perform, and why, will give the reader full knowledge to understand the evaluation sections described later on. Furthermore, an extensive setup section cover which tools were used to create the experiment enviroment used througout this project.

Section 4, latency performance evaluation, presents the methods used to enable the evaluation described in more depth in section 3. The results from the evaluation presents full insight in how well the CCN-lite application perform in the network. Parts of the results here, is to be viewed as necessary background knowledge in order to understand the upcoming feasibility evaluation in the next experiment.

Section 5, suitability evaluation, a method to following the data creation at the sensor is presented and implemented. The results provide argument to state that, with the correct software, CCN is a suitable replacement of MQTT or other publish/subscribe systems to retreive data from a IoT device. 

In section 6 the results from the previous two sections are analyzed and discussed in order to reach the conclusion presented in section 7.