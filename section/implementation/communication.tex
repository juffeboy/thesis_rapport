\subsubsection{Communication between gateway and sensor}
%In this experiment, a sensornode is connected via USB to a computer where one can monitor messages from the sensor. Through a 802.15.4 radio network, the sensor connect to a border router with Sparrow software running on it. All communication and message passing will be made between the the gateway and the sensornode over the 802.15.4 network. Above the 802.15.4 radio in the networking stack, data is encapsulated into 6LowPAN packets containing a full IPv6 header(of size 40 bytes). There are possibilites in Contiki-OS to compress the IP and networking headers by different strategies, but in in the implementation covered in this thesis, only the uncompressed 40 bytes IPv6 header will be considered. Thereafter the application data is encapsulated by either UDP or ICMPv6 as ther transportation protocol, both of those headers consist of 8 bytes. \todo{Behöver utvecklas för att ta med hela flödet, från application på GW till APP på sensor.\\\\ 

When the border router communicate with the sensor, all technologies, hardware and software described in earlier sections come together and cooperate.

When a CCN peek request is issued, it starts at the CCN-lite application and becomes detected at the Sparrow border router software. The router will transmit the request using the slip radio toward the sensor. At the sensor, the request will be responded or discarded. When the sensor responds with data, it will go on the same route backwards to the process on the router that initially made the CCN peek request.

All ping, CCN peek and data messages are sent through the 802.15.4 radio network. The sensor connects to the border router with the Sparrow software running on it. All the communication and message passing between the gatewey and the sensor node is made over the 802.15.4 network. Above the networking stack, the data is encapsulated into 6LowPAN packets, which contain a full IPv6 header (of size 40 byte). In Contiki OS, there are various header compression strategies for IP and UDP. In this thesis, all those compression techniques are turned off. Only the uncompressed 40 bytes IPv6 header is considered in this project. Thereafter, the application data is encapsulated by either UDP or ICMPv6 as their transport layer protocol. Both of those headers consists of 8 bytes. 


 



