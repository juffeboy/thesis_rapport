\subsubsection{Hardware}
The hardware setup used in this thesis consists of Texas Instrument CC2650 SensorTags, Zolertia Firefly and Raspberry Pi3. The sensortag produce sensor data that represent a IoT device. It communicate wirelessly with the Firefly-radio slip that is mounted on the Raspberry Pi3 which acts as a border gateway in this project. 

\paragraph{Texas Instrument CC2650 SensorTag}
The CC2650 sensortag is a wireless microcontroller developed by Texas Instruments \cite{CC2650}. The device is low cost, ultralow power device using the 2.4 Ghz radiofrequency to communicate with technologies such as 6LowPAN, Bluetooth and Zigbee. Due to its ultra low power consumption, the sensor can be powered by battery.
The CC2650 device contains a 32-bit ARM Cortex-M3 processor running at 48 Mhz, accompanied by 8 KB of cache and 20 KB of SRAM. It contains a total of 128 KB of programable flash memory, which can be used for different application system such as the Contiki-OS system. The sensor controller supports the measurement of different types of sensor data such as temperature readings, optical light values and more. 


\paragraph{Zolertia Firefly Slip radio}
The Firefly radio slip is developed by Zolertia \cite{Firefly}. The radio slip provides a network infrastructure for the IoT devices, enabling them to communicate efficiently through the air. The Firefly has great routing capabilities due to its support for several communication technologies, among them IEEE 802.15.4/6LowPAN and Zigbee. Another advantage using the Firefly is that it supports multiple types of frequency bands such as 2.4 GHz, 915- and 920 MHz band. Radio parameters such as modulation, data rate and transmission power are highly configurable.

\paragraph{Raspberry Pi 3}
The Raspberry Pi 3 is a single board computer developed by the Raspberry Pi Foundation \cite{RP3}. It contains components suchs as WIFI, several USB ports, 1 GB RAM and a quad core ARM processor among several other features. Due to its relativly high performance for a low price, it has become a popular developing tool used in projects at home, in school and for academic research.