\subsubsection{Software}
%There are some software running on the hardware in order to make everything to work. \todo{Rewrite all this. Arugment why we need this software on the hardware. Also describe very shortly the ccn application software. Also refer to background.}
The software setup used in this thesis consists of two CCN-lite applications \ref{CCN-LITE} \ref{yanqui}, Sparrow \ref{Sparrow} and Ping6. The CCN-lite software is used on the Linux platform and the portation described in previous section is used together with Contiki OS on the sensor node. To make them communicate with each other the Sparrow software creates a gateway together with the slip radio on the Raspberry Pi. This enables a reliable wireless communication channel between the router and the sensor.

\paragraph{Gateway CCN application}
%Has the possibility to send CCN interest requests and recieve data them. A peek latency tool has been developed to get the peek times.
%This application is sending the interest request through the gateway described in previous paragraph.
The CCN-lite application used on the border gateway has the possibility to send \textit{interest} message and recieve the \textit{data} for the request it has sent. To issue an \textit{interest} request, \textit{the ccn-lite-peek} application is used. It is an utility tool within the CCN-lite software that can create, encapsulate and send out requests on the radiolink and also recieve the corresponding data packet. A CCN peek application runs only one time, a request either receives the data or times out after a given time, thereafter the application terminates.\\\\
Additional functionality has, in this thesis, been added to this software to make it possible to calculate the latency for the difference between an \textit{interest} has been sent and when a response has been received. Throughout this thesis, CCN peek and peek is used interchangably in order to describe the round trip time from initiating a request to the time when that data has been received. 

\paragraph{Sensor CCN application}
The CCN application used for the sensors has a simple structure. Once the sensor has booted Contiki with CCN, it starts listening for incoming \textit{interest}-requests from the network and produces content objects.\\\\
When an \textit{interest} message is received at the sensor, a lookup in the cache will be performed to see if there is any matching content available. If there is a match in the \textit{content storage}, then a data message will be created with the content and sent int response towards the issuer. Otherwise, the \textit{interest} will be recorded in an entry in the \textit{pending interest table} and there is no response toward the user until the requested data has been produced.
An advantage is that this provides a possibility to ask for data that has not yet been produced. 
A consumer could potentially issue an \textit{interest} message a short time before the \textit{data} has been produced, which would firstly get the data directly to the user, and secondly reduce the latency on the network. 
\\\\
In every period a new content object is created. This object is cached into the \textit{content storage} to be retreived later on by an \textit{interest} request. There is also a lookup in the PIT to see if the newly created object already has an pending \textit{interest} request. If so, the object will be sent out towards the issuer and the request is to be considered consumed. Once the storage is full, the content will be removed from the \textit{content storage} in a FIFO-queue fashion.


\paragraph{Ping6}
The command ``live'' tool Ping6 is a utility software for linux systems which use the ICMPv6 protocol to send data over the network. In this thesis, it is used as a reference point for latency times in comparision to the CCN application.

