\subsubsection{Retrieve data}
In all experiments, the sensor node is connected via USB to a computer in order to retrieve measurement values. The live command tool TTY captures all these messages from the console that the sensor produces and make them readable for a user. 

After a few experiments, the result showed that the amount of information printed on the console had an impact on the latency.
After several iterations, a decision was made that there would be two versions of printing. One printing only a minimum of information and the other one printing all the essential information the evaluation needed. 

For verification purposes, the minimum printing sends information about whether an interest was responded to or not. 
For the essential printing, information regarding \textit{interest} arrival times, \textit{data} departure times and other metric values essential to the result were added. As a consequence, there is a higher consumption of processing power at the sensor. This is shown in later measurement results when minimum printing takes one tick on the clock (1/128th seconds) longer.
