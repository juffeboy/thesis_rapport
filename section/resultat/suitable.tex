\section{Suitable evaluation test}
\subsection{Purpose}
Follow sequence and periodicity from sensor
The purpose of this experiment is to create a suitable software that follows the periodicity of the data created by a sensor node.

The interval time between the data created by the sensor is known beforehand from the requestor.


Anledningen till att vi vill göra detta test är för att se om det är möjligt att använda sig av ICNs pull strategi för att hämta data, istället för en klassiskt publish subscribe system som beskrivs under MQTT. Är det möjligt att klienten kan erhålla datat så snabbt som möjligt det har tillverkats från sensorn. 

The purpose of this experiment is to evaluate the feasibility of the `one-time subscription'-model described by Ahlgren et al in [ref to paper]. How to retrieve a stable 



\begin{itemize}
\item retreieve data ass soon as it has been possible.
\item one time subscription
\item long time interval

\item refer to MQTT difference
\item CCN.
\end{itemize}
The purpose of this experiment is to evaluate the feasibility of the `one-time subscription' approach decsribed by Ahlgren et al [ref?]. 
The purpose 

The purpose of this experiment is to evaluate the feasibility of retrieving data from a producer as soon as it has been created. The result show 


The purpose of this experiment is to investigate the possibility for a client to retreive produced data from a sensor as soon as it has been created over a regular basis. The result show if such a possibility exists with data produced with a constant interval over a long period of time. 
The communication patterns of ICN and CCN is by natural pull-based, which is in contradiction to a regular publish/subscribe protocol such as MQTT. When using a publish/subscribe communication protocol, it can send the newly produced data directly towards a consumer. 

With the CCN possibility of storing \textit{interest} requests at the sensors, combined with the `one-time' subscription approach, it could be possible to achive the same functionallity with ICN. It is interesting to see how such a combination would perform over time. 



The purpose of this experiment is to measure the roundtrip time, latency, between a sensornode and a border gateway using ping and CCN peek commands. The results show which of these alternatives has the lowest latency, how much they differ and if there is a common pattern between them. It is also interesting to see how much time it takes for a sensor node to consume an CCN interest. From a more overview perspectrive, it is very important that the processing time of a CCN interest does not take to long time or to much resources and that it should be feasible for a sensor to deal with. If the computation time of returning data is to large, then CCN would be considered not suitable to be used for a IoT device. From here on, latency and round trip time is used interchangeably aswell as CCN peek and peek.

A regular publish/subscrieb protocol pushes the data from the producer to a broker to 


The purpose of this experiment is to investigate the ability for a client to retrieve the produced data from a sensor as soon as it has been created. 


\subsection{Delim}
It is not a purpose to find the latest sequence number, this is given at first hand. There are several different types of techniques to get the latest sequence number, but it is out of scope of this paper. 


\subsection{Method}
Develop software

\begin{itemize}
\item Follow interval on sensor, becomes a problem due to the fact that there is two different clocks. One on sensor that is producing at a certain interval speed, and one at the client that wants to consume at a regular interval speed.
\item Come to the right sensor
\item First timeout, usage of PIT, one time subscription model. Describe the one time subscription model.
\item After timeout, Algoritm usage to follow/correct


\item The algoritm.\\
The algorithm, shown in figure \ref{fig:onetime}, used in this project is a first trial in order to try to retreive the data with a one time subscription approach described earlier. 
The algorithm tries to fetch data at a certain interval in order to recieve the data as soon as it has been created on the sensor. It is calculates when to send the next request, towards the sensor, based by the round trip time of the current interval. It borrows the smoothed round trip time (srtt) technic from TCP, which makes the system more tolerent if one latency value is varying a lot from the normal. One must set a round trip target (rtt$\_$target) that the algortihm should try to follow, the target should be a factor of the minimum round trip time (rtt$\_$min). In the algortihm, there is a background correction factor that has to be calculated at every interval. This correction time is the difference between the srtt and the rtt$\_$target, and represent the difference in time between the two systems. It is added to the time it should send the next request, together with the interval time of the periodicity.



\end{itemize}

\begin{itemize}
\item Algortim. \\
The algorithm that was created for the consumer works as follows. 
It is shown in \ref{fig:onetime}
\item diff rttmin rtt$\_$target\\
	When using the pit, one can not retrieve the same latency times as in the previous measurement where the data was replied directly upon request. Instead, one must set a round trip target that is related to the minimum round trip time. The difference between rtt$\_$target and rtt$\_$min defines how large the time span is that we can regulate. The smaller differences, the more vulnerable the system becomes for variations of rtt. The minimum difference must be greater than the differences between the two intervals between the sensor and the gateway, if it gets less the algortihm stops working properly.

	The reason of using the rtt$\_$target is that we want to send the \textit{interest} request at a confortable distance before the data has been created.
	Instead, the request must be initiated before the 

\item SmoothRTT \\
	In order to calculate when to send the next request towards the sensor, one could use the rtt and subtract it from the interval. 
	However, this makes the system more sensitive to changes in the latencies which could lead to timeouts for a consumer. To solve this and make the latencies more stable over time, the smooth rtt is introduced. It uses a factor, $\alpha$, to decide how the current value should be weighted in comparison to the previous values. The greater $\alpha$ becomes, the importance the older values has and vice versa.
	If the srtt becomes to long, then we are to early initiating the request. At the same time, if the srtt becomes to short, then we are to late initiating the request.

\item Corr \\
	The difference in clock time between the sensor and the gw is measured with corr.
\end{itemize}

   
\begin{figure}
\begin{algorithm}[H]
 next = reference time\;
 rtt$\_$min\;
 rtt$\_$target = rtt$\_$min * \textit{x}\;
 $\alpha$ = 0 <= $\alpha$ <= 1\;
 \While{infinity}{
  rtt = send$\_$interest$\_$receive$\_$data\;
  \If{not timeout}{
   srtt = $\alpha$ $\times$ rtt + ($\alpha$ - 1) $\times$  srtt\;
   corr = srtt - rtt$\_$target\;
   }
   next$\_$time = next$\_$time + interval$\_$time + corr\;
   sleep(next$\_$time - current$\_$time)\;
 }

\end{algorithm}
\caption{Algorithm that makes it possible for a consumer to follow the creation of data with a certain interval.}
    \label{fig:onetime}
\end{figure}


\subsection{Result}
Suitable.

\begin{itemize}
\item Show that the algorithm can follow the interval and period over time. (show this with standard case 0.1 first.)\\
\end{itemize}
The results regarding latencies when using $\alpha$ = 0.1 and \textit{rtt$\_$target} = 2.5 is shown in figure[0.1], where the interval sequence number is shown on the x-axis and the roundtrip latency is shown on the y-axis. The same axis layout holds for figures[all figures], all of which regards the roundtrips from the consumers perspective. 
The results, illustrated in figure [fig], shows that the latency is stable between 90-100 ms, the srtt is stable with a small varians around 95 ms and that the corr factor is stable around 5 ms aswell. This results indicate that the consumer can retrieve the data in a stable manner.\\\\
The corresponding time from a sensors perspective is shown in figures [show all figures], where the interval sequence number is shown on the x-axis and the age of the data is shown on the y-axis. Since the sensor is making use of the PIT, the y-axis show the time between it recieved the \textit{interest} til the \textit{data} was created and responded towards the requestor. Figure [0.1] show that the age of the \textit{interest} is very stable at 62 ms (8 tick). The movement few movements up and down has to do with 
[Write about that this is indicating that it works!]


\begin{itemize}
\item Show difference between using $\alpha$ = 0.1, 0.9 and 1.
\end{itemize}
When $\alpha$ is changed to 0.9, the resulting curves, illustrated in figure [gw0.9] and [s0.9], shows an more alternating form. It alternates between values from 87 ms up to 117 ms in latency and the age of the \textit{interests} at the sensor is pending between 56 ms and 70 ms. This is due to the fact that the srtt has less importance and influence as $\alpha$ grows, and the time when next \textit{interest} is sent is more dependent on the latency time of the previous one. 
When $\alpha$ is set to 1, illustrated in figure [gw1] and [s1], the latency times alternate between 87 ms and 117 ms as well as in $\alpha$ = 0.9, but here the frequency is greater. In this case, the smoothing is not available and therefore the correction is only dependent on the last roundtrip time. 



\subsubsection{Different intervals}
\begin{itemize}
\item Show that it can handle diff in the interval. The producer has a different interval than the consumer. Works great without problems for positive changes, where the producer is creating the data at a slower interval, say 1.2 second per data.
\item Show difference in time. 
\end{itemize}
All of the previous results was made under the assumption that the two intervals are relative the same, one interval second on the sensor is almost the same as one interval second on the GW. To illustrate the functionallity when the sensor is providing data at different periodicity than the GW, other tests has been made.\\\\
In figure [gwdrift + 2$\%$] and [sdrift + 2$\%$], the sensor is creating the data with an interval of one second + 2$\%$ and the gw still use an interval period of one second. 
The results shows that the system is still stable over time, even though here is a positive drift in time on the sensor. The time it takes for an \textit{interest} to be consumed by the sensor is around 78 ms. The latency from the gw is overall higher, 105 ms to 125 ms, in comparission to without any difference in the time intervals, but the variation of srtt is still small and at the same levels. This is logic since the \textit{interest} is sent earlier from the GW towards the sensor, but at the the same intervals due to the algorithm. There is a cap though, when the \textit{interests} starts to timeout, the system will stop working properly and no data will be received at the gw. The drift on the sensor can not be greater, compared to the gw, than the timeout that the gw is setting on the \textit{interest}-packet when they are issued. Furthermore, the srtt is always greater than rtt$\_$target which provide stability in the system.
\begin{itemize}
	\item show diff in time negative, shorter time.
\end{itemize}
The system keeps beeing stable although the sensor period interval is less, -2$\%$, compared to the interval of the gw. The roundtrips are shorter than when the two intervals are the same and the age of the \textit{interests} are smaller too. The majority of the latencies is between 65 to 75 ms, as seen in figure [gw98], and the time it took to consumed the interest was around 40 ms as seen in figure [s98]. The reason why this is still stable is because the difference between rtt$\_$target and rtt$\_$min, 88 ms - 38 ms = 50 ms, is greater than the negative drift of -2 $\%$, approximatly 23 ms, of the sensors interval clock. Once the drift is greater than the difference between rtt$\_$min and rtt$\_$target, the system becomes unstable.






