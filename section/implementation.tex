\section{Implementation}

In order to evaluate the two major fields stated in the problemstatement, there are two thigns that has to happen.


\begin{itemize}
\item Utveckla problem statement

\item This is the problem
\item Continue more in depth than problem statement in chap 1.
\item That leds us to these experiments and why theyre done.
\end{itemize}
fakpweofkpwaef
\begin{itemize}
\item These are the problems that we want to solve.
\item Thats why we have developed these experiments.
\item For one, the latency difference between ping and peek evaluation.
\item Second, A evaluation regarding the algortim that makes it possible to `Deliver the data with as low latency possible from the sensor to the applicant.'


\end{itemize}

The thesis is therefore devided into two major experiments. First, in section [performance], latency performance experiments between ping and CCN peek are evaluated. [Continue write about why we need the performance evaluation.]
Secondly, experiments around the feasibility evaluation
Secondly, an evaluation regarding the feasibility experiments is made in section [feasibility]. 

Thatis why we have developed two major experiments, first concentrating on the latency performance evaluation between ping and CCN peek. The second experiment is made to evaluate `how deliver low latency '

ICN is by nature a pull paradigm where a consumer has to initiate a request of a particular data object in order to retrive it. This is in constrast to several IoT systems where the publish/subscribe approach is more common. An example of such a system is MQTT [3] where the producer sends the newly created data towards a broker, which then pushes the data to a user/consumer. By evaluating the feasibility, a major goal is to investigate if it is possible to achive similar outcome with ICN together with a onetime subscription model and if such a system would be stable. With the onetime subscription model, a consumer tries to pull the data from the producers by initiating a request just before the data has been created.


The theses tries to answer the question if ICN/CCN together with a one time subscription approach, could lead to that the latency in time from when the data is created is minimized, or as small as possible, equivivalent to MQTT. 

In the problem statement, 

The question the thesis tries to answer is wheather or not the client-server model, as in MQTT or similar, could be replaced with the information-centric model, and CCN, instead. The problem with the ICN model in IoT domain, lays in the nature of ICN with its data driven communication paradigm. A data will only pass through the network if it has been requested. In a normal network approach, a creator of a data can push its data toward a user. This is impossible with the ICN paradigm. 

The problem remains from a user/consumer perspecitve. It wants the data from the sensors as soon as it has been created, it does not matter wheater how it is done, but that it is done. 

In order to retrieve the data as soon as possible, a couple of experiments has to be evaluated in order to see if CCN could fit. That's why we have done two major experiments. one conserning the latency approach with CCN, A comparision between the CCN and ping is evaluated to see if ccn have large compuattion times that could harm the time it takes to process a request. 
secondly, experiments regardign how to retrieve data as soon as it has been produced in a periodically manner is discovered.



In order to invesigate wheather or not it is possible, the




\subsection{Setup}

Some info about why this is the setup. Maybe because of the previous section.
\begin{itemize}
%\item Hardware, copy from current background.
%\item Software, CCN application for SN and GW and write about ping6.
%\item Communication flow between gateway and sensor. EVERYthing thats happening then, from application layer from gw to sn.
%\item Debugging/printing to collect information.
\item Ingress.
\item Create a picutre that connects the dots, refere to this picture in the subsections.
%\item delim in setup (mayBE)
\end{itemize}


\subsection{Hardware}

\subsubsection{Texas Instrument CC2650 SensorTag}
The CC2650 sensortag is a wireless microcontroller developed by Texas Instruments \cite{CC2650}. The device is low cost, ultralow power device using the 2.4 Ghz radiofrequency to communicate with technologies such as 6LowPAN, Bluetooth and Zigbee. Due to its ultra low power consumption, the sensor can be powered by battery.
The CC2650 device contains a 32-bit ARM Cortex-M3 processor running at 48 Mhz, accompanied by 8 KB of cache and 20 KB of SRAM. It contains a total of 128 KB of programable flash memory, which can be used for different application system such as the Contiki-OS system. The sensor controller supports the measurement of different types of sensor data such as temperature readings, optical light values and more. 


\subsubsection{Zolertia Firefly Slip radio}
The Firefly radio slip is developed by Zolertia \cite{Firefly}. The radio slip provides a network infrastructure for the IoT devices, enabling them to communicate efficiently through the air. The Firefly has great routing capabilities due to its support for several communication technologies, among them IEEE 802.15.4/6LowPAN and Zigbee. Another advantage using the Firefly is that it supports multiple types of frequency bands such as 2.4 GHz, 915- and 920 MHz band. Radio parameters such as modulation, data rate and transmission power are highly configurable.

\subsubsection{Raspberry Pi 3}
The Raspberry Pi 3 is a single board computer developed by the Raspberry Pi Foundation \cite{RP3}. It contains components suchs as WIFI, several USB ports, 1 GB RAM and a quad core ARM processor among several other features. Due to its relativly high performance for a low price, it has become a popular developing tool used in projects at home, in school and for academic research.

\subsubsection{Software}
There are some software running on the hardware in order to make everything to work. \todo{Rewrite all this. Arugment why we need this software on the hardware. Also describe very shortly the ccn application software. Also refer to background.}

\paragraph{Gateway CCN application}
Has the possibility to send CCN interest requests and recieve data them. A peek latency tool has been developed to get the peek times.
This application is sending the interest request through the gateway described in previous paragraph.

\paragraph{Sensor CCN application}
The CCN application used for the sensors has a simple structure. Once the sensor has booted Contiki with CCN, it starts listening for incoming \textit{interest}-requests from the network and to produce content objects.

When an \textit{interest} message is received at the sensor, a lookup in the cache will be performed to see if there is any matching content available. If there is a match in the \textit{content storage}, then the data will be responded toward the issuer. Otherwise, the \textit{interest} will become an entry in the \textit{pending interest table} and no respond toward the user until the requested data has been produced.

In every period a new content object is created. This object is cached into the \textit{content storage} to be retreived later on by an \textit{interest} request. There is also a lookup in the PIT to see if the newly created object already has an pending \textit{interest} request. If so, the object will be sent out towards the issuer and the request is to be considered consumed. Once the storage is full, the content will be removed from the \textit{content storage} in a FIFO-queue fashion.


\paragraph{Ping6}

\subsubsection{Communication flow between gateway and sensor}
In this experiment, a sensornode is connected via USB to a computer where one can monitor messages from the sensor. Through a 802.15.4 radio network, the sensor connect to a border router with Sparrow software running on it. All communication and message passing will be made between the the gateway and the sensornode over the 802.15.4 network. Above the 802.15.4 radio in the networking stack, data is encapsulated into 6LowPAN packets containing a full IPv6 header(of size 40 bytes). There are possibilites in Contiki-OS to compress the IP and networking headers by different strategies, but in in the implementation covered in this thesis, only the uncompressed 40 bytes IPv6 header will be considered. Thereafter the application data is encapsulated by either UDP or ICMPv6 as ther transportation protocol, both of those headers consist of 8 bytes. \todo{Behöver utvecklas för att ta med hela flödet, från application på GW till APP på sensor.}
\\\\

\subsubsection{Retrieve data}
In all experiments, the sensor node is connected via USB to a computer in order to retrieve measurement values. The live command tool TTY captures all these messages from the console that the sensor produces and make them readable for a user. 

After a few experiments, the result showed that the amount of information printed on the console had an impact on the latency.
After several iterations, a decision was made that there would be two versions of printing. One printing only a minimum of information and the other one printing all the essential information the evaluation needed. 

For verification purposes, the minimum printing sends information about whether an interest was responded or not. 
For the essential printing, information regarding \textit{interest} arrival times, \textit{data} departure times and other metric values essential to the result were added. As a consequence, there is a higher consumption of processing power at the sensor. This is shown in later measurement results when minimum printing takes one tick on the clock (1/128th seconds) longer.
