\section{Experiment implementation/setup}
The experiment testbed used for this project includes CCN-lite code implemented on Contiki-OS for the sensor nodes. Slip-radios is based on Raspberry Pis to act as border routers. Suitable communication software described in previous sections enables the transmitting of data between the border router and the sensor nodes. The topology of the network is to be considered a star network. This experiment layout, illustrated in figure [insert figure], can be viewed a normal communication scenario between a node and its border router.


\begin{itemize}
	\item Show setup,
	\item A picture,
	\item how the program flows.
	\item imp
	The experimental implementation used in this paper includes CCN-lite application software running on Contiki-OS.
	The experimental implementation used in this paper can be divided into a two sections, the sensor and the border gateway.
	The sensor is runned by a modified CCN-lite application software that is running on Contiki-OS. 

	\item Software used for Mote
	\item Mote hooked up on computer
	\item Sparrow
	\item Raspberry Pi, Software used on Computer
	As a border gateway, a Raspberry Pi3 
	A Raspberry Pi3 is used together with a slip-radio of which fully support 
	As a border gateway, a Raspberry Pi3 is used together with a Zolertita[look up real name] slip radio. The slip radio fully supports communication over the 802.15.4 radio network. Together, 

	\item communication flow.


	\item sensortags
	The IoT device used in this project is the CC2650 described in previous section. The node has a Contiki-OS version of the CCN-lite code 

	The node has a CCN-lite implementation suitable for the Contiki-OS installed on it, providing the communication pattern suitable for this project.

\end{itemize}

\subsection{Delimitations}
Troughout this paper and the thesis project overall, power consumption has not been taken into consideration whatsoever. 

