\section{Experiment implementation/setup}
The experiment testbed used in this project includes CCN-lite application software runned on both Contiki-OS and on Linux. Firefly Slip-radios is based on Raspberry Pis to act as border routers. Suitable communication software described in previous sections enables the transmitting of data between the border router and the sensor nodes. The topology of the network is to be considered a star network. In this experiment layout, illustrated in figure [insert figure], the communication scenario is between a node and its border router.



\begin{itemize}
	\item Show setup,
	\item A picture,
	\item how the program flows.

\end{itemize}
\subsection{sensor}
The IoT device used is the CC2650 described in previous section. The node run a modified version of the CCN-lite implementation described in section [ref].

\subsection{gateway}
The gateway used is the Firefly Slip radio together together with a Raspberry Pi3. 


\subsection{Communication flow between gateway and sensor}
Kanske kapa dirr från performance setups första stycke?

\subsection{Sensor hooked up to computer}
för att läsa av värdena på sensorn, är den uppkopplad till datorn. Där kan man över USB läsa av alla debugg värden.

\subsection{Gateway CCN application}
Has the possibility to send CCN interest requests and recieve data them. A peek latency tool has been developed to get the peek times.
This application is sending the interest request through the gateway described in previous paragraph.

\subsection{Sensor CCN application}
The CCN application used for the sensors has a simple structure. Once the sensor has booted Contiki with CCN, it starts listening for incoming \textit{interest}-requests from the network and to produce content objects.

When an \textit{interest} message is received at the sensor, a lookup in the cache will be performed to see if there is any matching content available. If there is a match in the \textit{content storage}, then the data will be responded toward the issuer. Otherwise, the \textit{interest} will become an entry in the \textit{pending interest table} and no respond toward the user until the requested data has been produced.

In every period a new content object is created. This object is cached into the \textit{content storage} to be retreived later on by an \textit{interest} request. There is also a lookup in the PIT to see if the newly created object already has an pending \textit{interest} request. If so, the object will be sent out towards the issuer and the request is to be considered consumed. Once the storage is full, the content will be removed from the \textit{content storage} in a FIFO-queue fashion.



\subsection{Delimitations}
Troughout the thesis project, power consumption has not been taken into consideration whatsoever. 




