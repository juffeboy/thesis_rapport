\subsection{Related work}
When Jacobson et al. publicated the paper \textit{Networking named content} in 2009 it sparked ideas of an alternative approach of communicating in contrast to current IP networking \cite{Jacobson2009}. They implemented a prototype which replaced IP with CCN in the network stack and proved that it could be a potential alternative for the future Internet.\\\\
Since then, several research papers have been published comparing different ICN alternatives and their potential benefits and trade-offs when implemented as a network service\cite{Ahlgren2012}. Studies prove that it could be a suitable replacement of current IP networking structures, but there is a need for more performance analysis and studies\cite{Ahlgren2012}\cite{Greek-ICN-networking-survey-2014}.\\\\
However, it was not until the last couple of years that the research community started to investigate the feasibility and applicability of using ICN in the IoT domain. Several studies, the majority being literature studies or theoretic in nature, have been conducted recently or are currently ongoing. There only seem to be two implementation studies available to date.\\\\
In a conceptual study conducted by Ahlgren et al. \cite{Ahlgreniot}, it is concluded that an advantage with ICN, is that the naming of data is independent of the device that produces it. The decoupling between a producer and consumer of data could improve performance in lossy networks.Challenges reside in the naming of data that is produced periodically over time where a major issue is to retrieve the \textit{latest} value in that sequence. Potential solutions could be to implement a \textit{one-time subscription}, where the request is stored in the cache at the node until the data becomes available\cite{Ahlgreniot}.\\\\
Another study by Amadeo et al. \cite{iotchop} argues ICN is by nature close to the IoT domain. They also conclude that there is a need of further investigations regarding if ICN should be implemented as an overlay of existing IP infrastructure, coexist with IP, or if it should be a replacement in the same manner as proposed in \cite{Jacobson2009}.\\\\
Baccelli et al. were first to port of CCN-lite to the IoT operating system RIOT \cite{icniotexpinwild}\cite{RIOT}. The project was the first trial of implementing ICN without any IP protocol in the IoT domain. They compared their CCN-lite implemententation and a regular 6LowPAN/IPv6/RPL approach and saw that there were several advantages using ICN over IP. Although they identified several areas where further work needs to be done, they argue that ICN is applicable in the IoT domain.\\
Another implementation of CCN for IoT devices was a thesis project conducted by Yanqui Wu at SICS/KTH\cite{yanqui}. He ported the CCN-lite functionallity into another IoT operating system, Contiki OS, and implemented the software as an middle-layer between the application- and the transport layer. Although some evaluation was done, there was no further investigation on how well CCN performs at the application layer. \\

%\subsection{Related work}
%When Jacobson et al. publicated the paper \textit{Networking named content} in 2009 it sparked ideas of an alternative approach of communicating in contrast to current IP %networking \cite{Jacobson2009}. They implemented a prototype which replaced IP with CCN in the network stack and proved that it could be a potential alternative for the %future Internet.\\\\
%Since then, several research papers have been published comparing different ICN alternatives and their potential benefits and trade-offs when implemented as a network service\%cite{Ahlgren2012}. Studies prove that it could be a suitable replacement of current IP networking structures, but there is a need for more performance analysis and studies\%cite{Ahlgren2012}\cite{Greek-ICN-networking-survey-2014}.\\\\
%However, it was not until the last couple of years that the research community started to investigate the feasibility and applicability of using ICN in the IoT domain. %Several studies, the majority being literature studies or theoretic in nature, have been conducted recently or are currently ongoing. There only seem to be two implementation %studies available to date.\\\\
%In a conceptual study conducted by Ahlgren et al. \cite{Ahlgreniot}, it is concluded that an advantage with ICN, is that the naming of data is independent of the device that %produces it. The decoupling between a producer and consumer of data could improve performance in lossy networks\todo{Behöver koppla named data till IoT-context - Namngivet %sensordata // bengt}.Challenges reside in the naming of data that is produced periodically over time where a major issue is to retrieve the \textit{latest} value in that %sequence. \todo{Prata mer om hur man hittar senaste värdet. // anders} Potential solutions could be to implement a \textit{one-time subscription}, where the request is stored %in the cache at the node until the data becomes available\cite{Ahlgreniot}.\\\\
%Another study by Amadeo et al. \cite{iotchop} argues ICN is by nature close to the IoT domain. They also conclude that there is a need of further investigations regarding if %ICN should be implemented as an overlay of existing IP infrastructure, coexist with IP, or if it should be a replacement in the same manner as proposed in \cite{Jacobson2009}.%\\\\
%Baccelli et al. were first to port of CCN-lite to the IoT operating system RIOT \cite{icniotexpinwild}\cite{RIOT}. The project was the first trial of implementing ICN without %any IP protocol in the IoT domain. They compared their CCN-lite implemententation and a regular 6LowPAN/IPv6/RPL approach and saw that there were several advantages using ICN %over IP. Although they identified several areas where further work needs to be done, they argue that ICN is applicable in the IoT domain.\\
%Another implementation of CCN for IoT devices was a thesis project conducted by Yanqui Wu at SICS/KTH\cite{yanqui}. He ported the CCN-lite functionallity into another IoT %operating system, Contiki OS, and implemented the software as an middle-layer between the application- and the transport layer. Although some evaluation was done, there was %no further investigation on how well CCN performs at the application layer. \\
%
