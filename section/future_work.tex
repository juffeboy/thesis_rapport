\subsection{Future work}
%To further explore the usage of ICN in IoT networks there are several interesting aspects to consider.
This thesis shows a subset of the performance and suitability of using ICN in IoT networks. There are several other interesting aspects to consider for future studies. 
In this paper, the communication pattern was between a gateway and a sensor. One of the more critical aspects is how well ICN would perform if the network load would increase and how it would impact the performance overall. 
In a normal usage pattern, there would likely be several users wanting to access the produced data. There are indications from this thesis that the sensor's processing power is too slow to handle several \textit{interest} messages and deliver responses towards multiple users. It would be interesting to see how the software performs on sensors that have greater processing capabilities than the ones used in this thesis.

The algorithm presented here gives a user the possiblity to retreive data that is sequentially produced at a fixed interval. The interval rate at when the producer creates data is assumed to be known in advance by the consumer. Although this interval does not have to be exact, a rough estimation has to be given, otherwise the consumer's algorithm will not work properly. %It would be of interest to see if it is possible to develop an algorithm that can handle larger variations in interval length when the data is created. 
It would be of interest to se if it is possible to develop an algortihm that can handle larger variations in interval length which is not dependent on parameters such as \textit{rtt$\_$target}.

%\begin{itemize}
%\item how to get closer and reduce the overall latency time from a consumers perspective. \\
%\end{itemize}
One of the main benefits with the presented algortihm resides in the non-existing overhead it takes from data creation until it reaches the consumer, minimizing this time close to zero. However, the round trip time it takes from a consumer's perspective is not optimized today. From its view, an \textit{interest} message potentially spends most time in the PIT during the round trip.
Although the \textit{rtt$\_$target} parameter is specifying how well in advance a user can send the \textit{interest} message, there are no deeper studies regarding how close/low this can be in order to achive a stable system. 

Yet another aspect that would be interesting ot investigate is how to handle the `tune-in' period. Now, client cuts the interval by a small factor each interval until it receives a timeout. Then it `knows' that it has started to use the \textit{PIT}, thereafter using the algorithm. This can make the starting period take several iterations depending on how long the interval is. There are several different approaches to handle situations like these, but there are no investigations regarding which approach is the best in this case.

From a greater perspective, there are some interesting aspects to consider for future studies. First and foremost, the energy consumption of nodes/the system. This topic has not been covered in this thesis in any way. In order to make ICN a prominent choice for usage in IoT networks, it has to prove to be energy efficient, thereby there has to be further investigations in how well it performs from a energy consumption perspective. MQTT has a general advantage here with its push mechanism, where the sensor can be in sleep mode until it needs to transmit the newly created data. Since ICN does not have the same features, other solutions have to be provided in order to have a small energy footprint. A potential power saving strategy for ICN could be by sending burts of sequences of \textit{interest} messages towards the sensor, stacking them up in the \textit{PIT}. Thereby potentialy becoming more energy efficient. 

%First and foremost, energy consumption. In order to make ICN a adequant solution as i.e MQTT 
%In order to make ICN a suitable choice for IoT networks, 
Another interesting aspect for evaluating is how well ICN performs in comparison to other communication protocols such as CoAP or MQTT within the IoT domian. Not only by evaluating and comparing performance metrics such as protocol overhead times, various of latencies, but also a more deeper investigation regarding how suitable it is to use these communication protocols. Where the other protocols, CoAP and MQTT, provides a long term subscription approach in order to provide consumers with latest data, ICN has to use some kind of short time subscription approach to do the same, potentially giving it more overhead compared to the others.
This could test the limits of the ICN architecture from several perspectives and push the development of ICN as a whole forward.


%\begin{itemize}
%\item Getting the latest sequence\\
%The data produced by IoT devices is normally stamped with a sequence number that indicates when in time the data has been produced. In order to find the latest sequence %available there 
%
%There are several ways to find the latest sequence number available at the sensor. Troughout this thesis, the assumption has been made that this is known in advanced. 
%
%\end{itemize}



