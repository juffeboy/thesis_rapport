\section{Conclusion}

%\begin{itemize}
%\item Summarize your contributions\\
%The contributions of this report is an algortihm that allows a %consumer to follow the data created, in a periodic interval, on a %sensor node. 
%The data is received in the shortest available time from 
%The time from when the data was created until it has been received %at the 
%
%\item Conclusions from the results
%\item Implications for the future 
%\item Be breif!
%
%\item Its a suitable system. Works flawlessly.
%\end{itemize} 

The aim of this study was to evaluate the applicability of ICN in the IoT domain and its performance compared to legacy solutions. It should investigate if CCN is a suitable candidate that potentially could replace for example MQTT as an communication paradigm.
The evaluation considered performance measurements of the CCN-lite application components, the overall latency at the system and the feasibility of using ICN in the IoT domain.
%The performance evaluation showed that CCN-lite application is successful in delivering a low latency between a sensornode and its gateway. 
%When using the CCN peek application to retrieve data from the sensors CCN-lite application, it performs only a few milliseconds slower compared to the ping counterpart.

The performance evaluation showed that CCN-lite can deliver data from a sensornode to its gateway with low latency. When using the CCN peek application to retrieve data from the sensors CCN-lite implementation, it performs only a few milliseconds slower compared to the ping counterpart which was used as a reference point in the measurements.

The optimization of the CCN-lite code at the sensor shows that part of the long round trip times experienced earlier was due to various debug messages that were printed on the console. Furthermore, the sensor was put into sleep mode for a short period of time for each \textit{interest} message it received. The optimization of the code resulted in a reduction of the round trip times with around 100 ms compared to before. From round trip times of 130 ms to less than 30 ms.

The suitability evaluation showed that ICN is successful in achieving message delivering with a low delay between the creation of a data item and its delivering. When using of CCNs \textit{PIT} and sending the \textit{interest} messages before the data has been created, CCN can eliminate unnecessary waiting time between data creation and sending it. Thereby achiving similar latencies as when using MQTT.

The algortihm presented in this thesis is a first attempt to fetch data periodically produced at a fixed interval using ICN. The algorithm proved to be successful in retreiving data over long periods of time, no timeouts occured during the tests. %In order for it to work properly, it needs a reference time, the length of the interval, 

As the experiment shows, the algorithm handles high variations in roundtrip times without any issues. It is because of the logics behind the smoothed round trip time that makes the latency values more or less varying. As $\alpha$ grows, the significance of historical rtt-values decreases. 

Based on the findings in this thesis, the algorithm handles variances in clock drift to a certain degree. When the data publishing interval at the sensor is greater than the requesting interval of the consumer, the system stays stable. The length of the \textit{interest} message's timeout, at the consumer is the only variable that then can cause the algorithm to become unstable. When the data producing interval at the sensor is negative compared to the consumer the algortihm handles a limited difference in interval length. The difference between the \textit{rtt$\_$target} and \textit{rtt$\_$min} has to be be greater than the drift between the consumer interval and the producer, otherwise the system will become instable.

The conclusion drawn in this thesis is that the communication paradigm ICN provides a suitable alternative, in comparision to MQTT, in the IoT domain. A consumer, which was represented as a border gateway, can receive produced data from the producer with the lowest possible latency when it uses the algorithm provided in this thesis. 

%In order for it to work properly, it needs a reference time, length of the interval, 
%It was successful in pulling data even though there was a high $\alpha$,
%Even though the $\alpha$ parameter was changed, which made the round trip times alternate more, the algorithm proved to be stable and successful in retreiving the data.

%Even though round trips alternated, sometimes because of high $\alpha$, the algorithm proved to stay stable and successful in retreiving the data. 
%Even though the round trip times variated, sometimes due to high $\alpha$, the 

%Även om rundtrip tiderna varierade, oftast på grund av ett högt satt alpha, kunde algorithmen parera för detta och leverera datat stabilt till användaren.


%\begin{itemize}
%\item Varfor det fungerar
%\item Fungerar bra for olika deklay pa klockan.
%\end{itemize}
%=======
%\begin{itemize}
%\item Total verdict
%TOTAL verdict.
%The two evaluations combined, have shown that it is feasible to use ICN as a communication paradigm in the IoT domian. 
%\end{itemize}
%
%\begin{itemize}
%\item kort om vad som har gjortst
%\item beskrvi performance evaluation
%\item beskriv feasibile evaluation
%\item beskrvi den mer i detalj.
%\item beskriv helheten att det fungerar bra om man kor det i bode kortare och langre tidsserier.
%\end{itemize}
\newpage
\subsection{Future work}
%To further explore the usage of ICN in IoT networks there are several interesting aspects to consider.
This thesis shows a subset of the performance and suitability of using ICN in IoT networks. There are several other interesting aspects to consider for future studies. 
In this paper, the communication pattern was between a gateway and a sensor. One of the more critical aspects is how well ICN would perform if the network load would increase and how it would impact the performance overall. 
In a normal usage pattern, there would likely be several users wanting to access the produced data. There are indications from this thesis that the sensor's processing power is too slow to handle several \textit{interest} messages and deliver responses towards multiple users. It would be interesting to see how the software performs on sensors that have greater processing capabilities than the ones used in this thesis.

The algorithm presented here gives a user the possiblity to retreive data that is sequentially produced at a fixed interval. The interval rate at when the producer creates data is assumed to be known in advance by the consumer. Although this interval does not have to be exact, a rough estimation has to be given, otherwise the consumer's algorithm will not work properly. %It would be of interest to see if it is possible to develop an algorithm that can handle larger variations in interval length when the data is created. 
It would be of interest to se if it is possible to develop an algortihm that can handle larger variations in interval length which is not dependent on parameters such as \textit{rtt$\_$target}.

%\begin{itemize}
%\item how to get closer and reduce the overall latency time from a consumers perspective. \\
%\end{itemize}
One of the main benefits with the presented algortihm resides in the non-existing overhead it takes from data creation until it reaches the consumer, minimizing this time close to zero. However, the round trip time it takes from a consumer's perspective is not optimized today. From its view, an \textit{interest} message potentially spends most time in the PIT during the round trip.
Although the \textit{rtt$\_$target} parameter is specifying how well in advance a user can send the \textit{interest} message, there are no deeper studies regarding how close/low this can be in order to achive a stable system. 

Yet another aspect that would be interesting ot investigate is how to handle the `tune-in' period. Now, client cuts the interval by a small factor each interval until it receives a timeout. Then it `knows' that it has started to use the \textit{PIT}, thereafter using the algorithm. This can make the starting period take several iterations depending on how long the interval is. There are several different approaches to handle situations like these, but there are no investigations regarding which approach is the best in this case.

From a greater perspective, there are some interesting aspects to consider for future studies. First and foremost, the energy consumption of nodes/the system. This topic has not been covered in this thesis in any way. In order to make ICN a prominent choice for usage in IoT networks, it has to prove to be energy efficient, thereby there has to be further investigations in how well it performs from a energy consumption perspective. MQTT has a general advantage here with its push mechanism, where the sensor can be in sleep mode until it needs to transmit the newly created data. Since ICN does not have the same features, other solutions have to be provided in order to have a small energy footprint. A potential power saving strategy for ICN could be by sending burts of sequences of \textit{interest} messages towards the sensor, stacking them up in the \textit{PIT}. Thereby potentialy becoming more energy efficient. 

%First and foremost, energy consumption. In order to make ICN a adequant solution as i.e MQTT 
%In order to make ICN a suitable choice for IoT networks, 
Another interesting aspect for evaluating is how well ICN performs in comparison to other communication protocols such as CoAP or MQTT within the IoT domian. Not only by evaluating and comparing performance metrics such as protocol overhead times, various of latencies, but also a more deeper investigation regarding how suitable it is to use these communication protocols. Where the other protocols, CoAP and MQTT, provides a long term subscription approach in order to provide consumers with latest data, ICN has to use some kind of short time subscription approach to do the same, potentially giving it more overhead compared to the others.
This could test the limits of the ICN architecture from several perspectives and push the development of ICN as a whole forward.


%\begin{itemize}
%\item Getting the latest sequence\\
%The data produced by IoT devices is normally stamped with a sequence number that indicates when in time the data has been produced. In order to find the latest sequence %available there 
%
%There are several ways to find the latest sequence number available at the sensor. Troughout this thesis, the assumption has been made that this is known in advanced. 
%
%\end{itemize}





