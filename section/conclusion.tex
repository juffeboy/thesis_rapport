\section{Conclusion}

%\begin{itemize}
%\item Summarize your contributions\\
%The contributions of this report is an algortihm that allows a %consumer to follow the data created, in a periodic interval, on a %sensor node. 
%The data is received in the shortest available time from 
%The time from when the data was created until it has been received %at the 
%
%\item Conclusions from the results
%\item Implications for the future 
%\item Be breif!
%
%\item Its a suitable system. Works flawlessly.
%\end{itemize} 

The aim of this study was to evaluate the applicability of ICN in the IoT domain and its performance compared to legacy solutions. It should investigate if CCN is a suitable candidate that potentially could replace for example MQTT as an communication paradigm.
The evaluation considered performance measurements of the CCN-lite application components, the overall latency at the system and the feasibility of using ICN in the IoT domain.
%The performance evaluation showed that CCN-lite application is successful in delivering a low latency between a sensornode and its gateway. 
%When using the CCN peek application to retrieve data from the sensors CCN-lite application, it performs only a few milliseconds slower compared to the ping counterpart.

The performance evaluation showed that CCN-lite can deliver data from a sensornode to its gateway with low latency. When using the CCN peek application to retrieve data from the sensors CCN-lite implementation, it performs only a few milliseconds slower compared to the ping counterpart which was used as a reference point in the measurements.

The optimization of the CCN-lite code at the sensor shows that part of the long round trip times experienced earlier was due to various debug messages that were printed on the console. Furthermore, the sensor was put into sleep mode for a short period of time for each \textit{interest} message it received. The optimization of the code resulted in a reduction of the round trip times with around 100 ms compared to before. From round trip times of 130 ms to less than 30 ms.

The suitability evaluation showed that ICN is successful in achieving message delivering with a low delay between the creation of a data item and its delivering. When using of CCNs \textit{PIT} and sending the \textit{interest} messages before the data has been created, CCN can eliminate unnecessary waiting time between data creation and sending it. Thereby achiving similar latencies as when using MQTT.

The algortihm presented in this thesis is a first attempt to fetch data periodically produced at a fixed interval using ICN. The algorithm proved to be successful in retreiving data over long periods of time, no timeouts occured during the tests. %In order for it to work properly, it needs a reference time, the length of the interval, 

As the experiment shows, the algorithm handles high variations in roundtrip times without any issues. It is because of the logics behind the smoothed round trip time that makes the latency values more or less varying. As $\alpha$ grows, the significance of historical rtt-values decreases. 

Based on the findings in this thesis, the algorithm handles variances in clock drift to a certain degree. When the data publishing interval at the sensor is greater than the requesting interval of the consumer, the system stays stable. The length of the \textit{interest} message's timeout, at the consumer is the only variable that then can cause the algorithm to become unstable. When the data producing interval at the sensor is negative compared to the consumer the algortihm handles a limited difference in interval length. The difference between the \textit{rtt$\_$target} and \textit{rtt$\_$min} has to be be greater than the drift between the consumer interval and the producer, otherwise the system will become instable.

The conclusion drawn in this thesis is that the communication paradigm ICN provides a suitable alternative, in comparision to MQTT, in the IoT domain. A consumer, which was represented as a border gateway, can receive produced data from the producer with the lowest possible latency when it uses the algorithm provided in this thesis. 

%In order for it to work properly, it needs a reference time, length of the interval, 
%It was successful in pulling data even though there was a high $\alpha$,
%Even though the $\alpha$ parameter was changed, which made the round trip times alternate more, the algorithm proved to be stable and successful in retreiving the data.

%Even though round trips alternated, sometimes because of high $\alpha$, the algorithm proved to stay stable and successful in retreiving the data. 
%Even though the round trip times variated, sometimes due to high $\alpha$, the 

%Även om rundtrip tiderna varierade, oftast på grund av ett högt satt alpha, kunde algorithmen parera för detta och leverera datat stabilt till användaren.


%\begin{itemize}
%\item Varfor det fungerar
%\item Fungerar bra for olika deklay pa klockan.
%\end{itemize}
%=======
%\begin{itemize}
%\item Total verdict
%TOTAL verdict.
%The two evaluations combined, have shown that it is feasible to use ICN as a communication paradigm in the IoT domian. 
%\end{itemize}
%
%\begin{itemize}
%\item kort om vad som har gjortst
%\item beskrvi performance evaluation
%\item beskriv feasibile evaluation
%\item beskrvi den mer i detalj.
%\item beskriv helheten att det fungerar bra om man kor det i bode kortare och langre tidsserier.
%\end{itemize}
\newpage
\subsection{Future work}
%To further explore the usage of ICN in IoT networks there are several interesting aspects to consider.
This thesis only showed a subset of the performance and suitability of using ICN in IoT networks. There are several other interesting aspects to consider for future studies. 


\begin{itemize}
\item Scalability
\end{itemize}
%The scalability aspect is 
\begin{itemize}
\item Interval changes on the sensor. Now the drift is constant. Would be interesting to see how it could handle a variated interval.
\end{itemize}



